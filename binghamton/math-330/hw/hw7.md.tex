\section{Do Project 5.3 using def. of set equality. Do Projects 5.5,
5.12(i,ii), 5.16(ii), 5.21(i), and 5.4A(a). Prove Prop. 5.20(ii). Do
Project
4.3.}\label{do-project-5.3-using-def.-of-set-equality.-do-projects-5.5-5.12iii-5.16ii-5.21i-and-5.4aa.-prove-prop.-5.20ii.-do-project-4.3.}

\begin{quote}
\textbf{Project 5.3.}
\end{quote}

Determine which of the following set equalities are true. If a statement
is true, prove it. If it is false, explain why this set equality does
not hold.

Definition of set equality: \[A=B \iff A\subseteq B \land B\subseteq A\]

\begin{enumerate}
\def\labelenumi{(\roman{enumi})}
\tightlist
\item
  \(D = E\), where\\
  \(D := \{3x : x \in \mathbb{N} \land x > 7\}\)\\
  \(E := \{y : y \in \mathbb{N}\}\)
\end{enumerate}

This statement is false. \textbf{\emph{Proof.}}\\
We follow a simple proof by contradiction - If \(D=E\), then
\(D\subseteq E\) and \(E \subseteq D\).\\
By definition, \(E \subseteq D\) means \(n\in E\implies n\in D\).\\
We know \(1\in\mathbb{N}\). Setting \(y=1\) in the definition of \(E\)
shows us that \(1\in E\).\\
By the definition of \(D\), \(D\) contains natural numbers of the form
\(3x\), where \(x\in\mathbb{N}\).\\
We know that \(1\) is not divisible by 3, and therefore cannot be
represented as \(3x\) for \(x\in\mathbb{N}\).\\
We see \(1\in E\), but \(1\notin D\). Therefore, \(E\subseteq D\) is
false, and consequently, \(E\neq D\).\\
\textbf{\emph{QED}.}

\begin{enumerate}
\def\labelenumi{(\roman{enumi})}
\setcounter{enumi}{2}
\tightlist
\item
  \(D = B\), where\\
  \(D := \{3x : x \in \mathbb{N} \land x > 7\}\)\\
  \(B := \{3y + 21 : y \in \mathbb{N}\}\)
\end{enumerate}

These sets are equal.\\
\textbf{\emph{Proof.}}\\
First prove \(D\subseteq B\), that is, \(n\in D \implies n\in B\)\\
Let \(n\in D\). We know \(n\) is a natural number with the form \(3x\),
for any \(x\in\mathbb{N}\) where \(x>7\).\\
Now, we prove \(n\) can be written in the form \(3y+21\), for any
\(y\in\mathbb{N}\).\\
\texttt{//\ COMMENT:\ Is\ there\ some\ better\ formal\ way\ to\ say\ what\ im\ trying\ to\ say\ here?}
Let \(y=x-7\) for any \(x>7\). By definition,
\((x-7\in \mathbb{N})\iff (x>7)\). Therefore, \(x-7\) can be used as a
substitution for \(y\).\\
Solving for \(x\), \(x=y+7\). We substitute \(y+7\) for \(x\):\\
\(n=3x=3(y+7)=3y+21\) \(\therefore n\in D \implies n\in B\).\\
\(\therefore D\subseteq B\).

Now prove \(B\subseteq D\), that is, \(m\in B \implies m\in D\).\\
Let \(m\in B\). \(m=3y+21\) for any \(y\in\mathbb{N}\).\\
Now, we prove this implies \(m=3x\), for some \(x\in\mathbb{N}\) where
\(x>7\).\\
\(m=3y+21=3(y+7)\).\\
We let \(x=y+7\), for any \(y\in\mathbb{N}\). This works because solving
for \(y\), \(y=x-7\), and \((x-7\in\mathbb{N})\iff (x>7)\).\\
With \(x=y+7\), we substitute:\\
\(m=3(y+7)=3x\) for all \(x>7\).\\
\(\therefore m\in B\implies m\in D\).\\
\(\therefore B \subseteq D\).

Since \(D\subseteq B\) and \(B \subseteq D\), then \(B=D\).\\
\textbf{\emph{QED.}}

\begin{enumerate}
\def\labelenumi{(\roman{enumi})}
\setcounter{enumi}{1}
\tightlist
\item
  \(C = G\), where\\
  \(C := \{x + 7 : x \in \mathbb{N}\}\)\\
  \(G := \{y : y \in \mathbb{N} \land y > 7\}\)
\end{enumerate}

These sets are equal.\\
\textbf{\emph{Proof.}}\\
First prove \(C\subseteq G\).\\
We need to prove \(n\in C \implies n\in G\).\\
Let \(n\in C\). We know \(n=x+7\), for any \(x\in\mathbb{N}\).\\
Now, we prove this implies \(n=y\) for some \(y\in\mathbb{N}\) if
\(y>7\).\\
Since \(y>7 \iff y-7\in \mathbb{N}\), we can let \(x=y-7\), where we
know \(x\in\mathbb{N}\).\\
\(y=x+7\), then we substitute:\\
\(n=x+7=y\).\\
\(\therefore n\in C \implies n\in G\).\\
\(\therefore C\subseteq G\).

Now prove \(G\subseteq C\).\\
Let \(m\in G\). We know \(m=y\), for any \(y\in\mathbb{N}\) if
\(y>7\).\\
Prove this implies \(m=x+7\), for some \(x\in\mathbb{N}\).\\
Following the reasoning from before, \(x=y-7\).\\
Substituting: \(m=y=x+7\).\\
\(\therefore n\in G \implies n\in C\).\\
\(\therefore G\subseteq C\).

Since \(C \subseteq G\) and \(G \subseteq C\), then \(C=G\).\\
\textbf{\emph{QED.}}

\begin{quote}
\textbf{Project 5.5.}
\end{quote}

When reading or writing a set definition, pay attention to what is a
variable inside the set definition and what is not a variable. As
examples, how do the following pairs of sets differ?

\(S := \{m : m \in \mathbb{N}\}\) and \(T_m := \{m\}\) for a specified
\(m \in \mathbb{N}\).\\
\(S\) and \(T_m\) differ because the definition of \(S\) is constructing
a set of all natural numbers, and \(T_m\) just contains one specified
element which is a natural number.

\(U := \{my : y \in \mathbb{Z}, m \in \mathbb{N}, my > 0\}\) and
\(V_m := \{my : y \in \mathbb{Z}, my > 0\}\) for a specified
\(m \in \mathbb{N}\).\\
In the set definition of \(U\), a number is being constructed for each
(pair of?) integer and natural number. In \(V_m\), \(m\) is a constant
defined outside of the set definition, so each element of \(V_m\) will
be a multiple of the predefined constant \(m\).

\(V_m\) and \(W_m := \{my : y \in \mathbb{Z}, y > 0\}\) for a specified
\(m \in \mathbb{Z}\).\\
The sets have the same definition and use the same constant \(m\), so
they are the same.

\begin{quote}
\textbf{Project 5.12.}
\end{quote}

Determine which of the following statements are true for all sets \(A\),
\(B\), and \(C\). If a double implication fails, determine whether one
or the other of the possible implications holds. If a statement is true,
prove it. If it is false, provide a counterexample.

\begin{enumerate}
\def\labelenumi{(\roman{enumi})}
\tightlist
\item
  \(C \subseteq A\) and \(C \subseteq B \iff C \subseteq (A \cup B)\).
\end{enumerate}

Only \(C \subseteq A\) and
\(C \subseteq B \implies C \subseteq (A \cup B)\) is true.\\
\textbf{\emph{Proof.}}\\
Let \(x\in C\). Since \(C\subseteq A \iff (x\in C \implies x\in A)\),
\(x\in A\).\\
By definition, \(A\cup B=\set{x:x\in A \text{ or } x\in B}\).\\
Since \(x\in A \text{ or } x\in B\) is always true if \(x\in A\), all
elements of \(A\) must be in \(A\cup B\), that is,
\(A\subseteq (A\cup B)\).\\
Additionally, since \(x\in A \text{ or } x\in B\) is always true if
\(x\in B\), \(B\subseteq (A\cup B)\) (this will be used later).\\
Since \(x\in A\) and \(A\subseteq (A\cup B)\), \(x\in (A\cup B)\).\\
We have shown \(x\in C \implies x\in (A\cup B)\), and this is precisely
the definition of \(C\subseteq (A\cup B)\).\\
\(\therefore C \subseteq A\) and
\(C \subseteq B \implies C \subseteq (A \cup B)\).

Now, we disprove \(C \subseteq (A \cup B) \implies C \subseteq A\) and
\(C \subseteq B\), by a counterexample.\\
We define the following sets:\\
\(A:=\set{1,2,3}\)\\
\(B:=\set{4,5}\)

\((A\cup B) = \set{1,2,3,4,5}\)\\
Define the set \(C:=\set{2,3,4}\), where \(C\subseteq (A\cup B)\).\\
\(C=\set{2,3,4}\) is not a subset of \(A=\set{1,2,3}\), which means
\((C \subseteq A\) and \(C \subseteq B)\) is false.\\
\textbf{\emph{QED.}}

\begin{enumerate}
\def\labelenumi{(\roman{enumi})}
\setcounter{enumi}{1}
\tightlist
\item
  \(C \subseteq A\) or \(C \subseteq B \iff C \subseteq (A \cup B)\).
\end{enumerate}

Only \(C \subseteq A\) or
\(C \subseteq B \implies C \subseteq (A \cup B)\) is true.\\
\textbf{\emph{Proof.}}\\
Let \(x\in C\). It is sufficient to prove two cases: when
\(C\subseteq A\) and when \(C\subseteq B\).\\
Let \(C\subseteq A\). Since \(x\in C\), \(x\in A\). We know from
previous results that \(A\subseteq (A\cup B)\). Since \(x\in A\),
\(x\in (A\cup B)\).\\
Now, let \(C\subseteq B\). Since \(x\in C\), \(x\in B\). From previous
results, \(B\subseteq (A\cup B)\). Since \(x\in B\),
\(x\in (A\cup B)\).\\
We have shown that when \(C\subseteq A\),
\(x\in C \implies x\in (A\cup B)\).\\
We have also shown that when \(C\subseteq B\),
\(x\in C \implies x\in (A\cup B)\).\\
Note that these implications are the same, so if \(C\subseteq A\) or
\(C\subseteq B\), \(x\in C \implies x\in (A\cup B)\), or rewritten,
\(C\subseteq (A\cup B)\).\\
\(\therefore C \subseteq A\) or
\(C \subseteq B \implies C \subseteq (A \cup B)\).

Now, we disprove \(C \subseteq (A \cup B) \implies C \subseteq A\) or
\(C \subseteq B\), by a counterexample.\\
We define the same sets as before:\\
\(A:=\set{1,2,3}\)\\
\(B:=\set{4,5}\)

\((A\cup B) = \set{1,2,3,4,5}\)\\
Define the set \(C:=\set{2,3,4}\), where \(C\subseteq (A\cup B)\).

\(C\subseteq A\) is false, and \(C\subseteq B\) is false, which means
\((C\subseteq A\) or \(C\subseteq B)\) is false.\\
\textbf{\emph{QED.}}

\begin{quote}
\textbf{Project 5.16}
\end{quote}

Someone tells you that the following equalities are true for all sets
\(A\), \(B\), \(C\). In each case, either prove the claim or provide a
counterexample.

\begin{enumerate}
\def\labelenumi{(\roman{enumi})}
\setcounter{enumi}{1}
\tightlist
\item
  \(A \cap (B - C) = (A \cap B) - (A \cap C)\).
\end{enumerate}

We rewrite the equality statement as follows:\\
\(A \cap \set{x : x\in B \land x\notin C} = \set{y : y \in A \land y \in B} - \set{y : y \in A \land y\in C}\)\\
\(\set{x : x\in A \land x\in B \land x\notin C} = \set{y : (y \in A \land y \in B) \land \lnot (y \in A \land y\in C)}\)\\
\(\set{x : x\in A \land x\in B \land x\notin C} = \set{y : (y \in A \land y \in B) \land (y \notin A \lor y\notin C)}\)\\
For the definition on the right side, we know \(y\in A\). Therefore,
\(y\notin A\) is always false.\\
\(\set{x : x\in A \land x\in B \land x\notin C} = \set{y : (y \in A \land y \in B) \land y\notin C}\)\\
\(\set{x : x\in A \land x\in B \land x\notin C} = \set{y : y \in A \land y \in B \land y\notin C}\)\\
These are exactly the same definitions. Therefore, the equality holds.

\begin{quote}
\textbf{Project 5.21}
\end{quote}

Let \(A\), \(B\), \(C\), \(D\) be sets. Decide whether each of the
following statements is true or false; in each case prove the statement
or give a counterexample.

\begin{enumerate}
\def\labelenumi{(\roman{enumi})}
\tightlist
\item
  \((A \times B) \cup (C \times D) = (A \cup C) \times (B \cup D)\).
\end{enumerate}

(1,3) (2,3) (1,4) (2,4)

This statement is false. We prove this with a counterexample:\\
Define the following sets:\\
\(A:=\set{1}\)\\
\(B:=\set{2}\)\\
\(C:=\set{3}\)\\
\(D:=\set{4}\)

\(A \times B = \set{(1,2)}\) \(C \times D = \set{(3,4)}\)
\((A \times B) \cup (C \times D) = \set{(1,2), (3,4)}\)

\(A \cup C = \set{1,3}\) \(B \cup D = \set{2,4}\)
\((A \cup C) \times (B \cup D) = \set{(1,2), (1,4), (3,2), (3,4)}\)

These two results are not equal.

\begin{quote}
\textbf{Functions: 5.4A(a)}
\end{quote}

\textbf{A function \(g:X\rightarrow Y\) is defined as follows:}

\begin{longtable}[]{@{}cc@{}}
\toprule\noalign{}
\(x\) & \(g(x)\) \\
\midrule\noalign{}
\endhead
\bottomrule\noalign{}
\endlastfoot
cat & 3 \\
squirrel & 0 \\
sunflower & 1 \\
grass & 4 \\
\end{longtable}

\textbf{What is the domain, \(\text{Dom}(g)\)?}

\(\set{cat, squirrel, sunflower, grass}\)

\textbf{What is the codomain, \(\text{Codom}(g)\)?}

The set \(Y\)

\textbf{What is the image, \(\text{Im}(g)\)?}

\(\set{3, 0, 1, 4}\)

\textbf{Is \(g\) injective (one-to-one)?}

Yes.

\textbf{Is \(g\) surjective (onto)?}

Yes. (Assuming \(Y=\set{3,0,1,4}\))

\textbf{Is \(g\) bijective (a one-to-one correspondence)?}

Yes. Injectivity and surjectivity \(\iff\) bijectivity.

\begin{quote}
\textbf{Proposition 5.20.}
\end{quote}

\begin{enumerate}
\def\labelenumi{(\roman{enumi})}
\setcounter{enumi}{1}
\tightlist
\item
  \(A \times (B \cap C) = (A \times B) \cap (A \times C)\)\\
  \textbf{\emph{Proof.}}\\
  Let the ordered pair \((a, x)\in A\times (B\cap C)\).\\
  \((a, x)\in A\times (B\cap C)\) \(\iff a\in A\) and
  \(x\in (B\cap C)\), by definition of the cartesian product.\\
  \(\iff a\in A\) and \(x\in B\) and \(x\in C\), by definition of the
  intersection operation.\\
  We now have the following true statement:
  \(a\in A \land x\in B \land x\in C\)\\
  \(\iff (a\in A \land x\in B) \land (a\in A \land x\in C)\)\\
  \(\iff [(a, x) \in (A \times B)] \land [(a, x) \in (A \times C)]\)\\
  \(\iff (a, x)\in [(A\times B) \cap (A\times C)]\), by definition of
  intersections.\\
  Since
  \((a,x)\in A\times (B\cap C)\iff (a,x)\in [(A\times B) \cap (A\times C)]\),
  by the definition of set equality, these two sets are equal.\\
  \textbf{\emph{QED.}}
\end{enumerate}

\begin{quote}
\textbf{Project 4.3} (The \(x + 1\) problem)
\end{quote}

We revise the \(3x + 1\) problem as follows: Pick your favorite natural
number \(m\), and define the following sequence:

\begin{enumerate}
\def\labelenumi{(\roman{enumi})}
\item
  Define \(x_1 := m\).
\item
  Assuming \(x_n\) defined, define \(x_{n+1} :=
  \begin{cases}
  \frac{x_n}{2} & \text{if } x_n \text{ is even,} \\
  x_n + 1 & \text{otherwise.}
  \end{cases}\)
\end{enumerate}

Does this sequence eventually take on the value \(1\), no matter what
\(m \in \mathbb{N}\) one chooses as the starting point? Try to prove
your assertion.

Uhh\ldots{} \texttt{unfinished}
